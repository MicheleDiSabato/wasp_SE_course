Overall, I agree with the distinction between data scientists and software engineers proposed in Chapter 1 of the book. Specifically, I agree on the fact that these two roles are very distinct and rely on different (academic) backgrounds. Because of this, in my opinion, it is challenging to be proficient in both “data science” and ”software engineering”, since both fields come with a broad collection of challenges. At the same time, the definition of the role of data scientists provided in the book seems to be simplified and generalized. Data scientists seem to be depicted as workers who are only focused on maximizing the accuracy of their model and who are almost unaware of the existence of a broader system that comes with other requirements. Instead, software engineers are described as individuals who work on almost all the rest of the software product. Nowadays, every data scientist most likely knows that the performance of their ML model is only one piece of a bigger picture, together with training costs and automation in the training/re-training pipeline. At the same time, it is clear that the author of the book uses the term “data scientist” to purposefully describe an outdated role, which is being replaced by ML engineers. \\
There is a common view of the data science pipeline that goes from data collection to model deployment. The author of the book argues that this is unrealistic, since data scientists must also deal with deployment and monitoring in an efficient, scalable, and robust way. These components inevitably influence the core data science tasks such as model training, validation, and feature engineering, and therefore cannot be considered in isolation. It would be unrealistic for a company to expect candidates (“unicorns”, using the book's terminology) who are both proficient in building models for their needs, and also highly skilled in deploying these models efficiently and at scale. Tools that help integrate these steps without requiring infrastructure to be built from scratch already exist, such as Hopsworks (\cite{Hopsworks}). However, every company’s needs are unique, and the MLOps part of the pipeline should therefore be tailored to the organization. This means that, in my opinion, the extended model pipeline\footnote{This is the pipeline described in Chapter 2, going from “model requirements” to “model deployment”} should be handled by two roles at the same time. Both roles should fall in the broader category of ML engineers but with different specifications. One role should have a strong background in building models, the other should also be an ML engineer but with a solid background in software engineering tailored to ML operations. Throughout the book, the authors describe a T-shaped employee, who is proficient in one specific topic in their field, but who also has broad general knowledge of other topics that pertain to the field. Both roles described previously should be T-shaped employees, being proficient in either model building, feature engineering, and data science, or MLOps. Their broad knowledge should allow them to communicate easily between each other regarding possible model choices or architecture choices to make models deployable. The separation of these two ML engineering roles allows for a more flexible interaction between the various teams who work on the product. Indeed, as mentioned in the book, the extended model pipeline is just one component of a broader system, which consists of the actual software architecture of the product and the interface with the external world (i.e., the users). This means that if MLOps are dealt with by someone who has a strong SE background, then it would be easier for them to communicate with the software developers who deal with this broader infrastructure. 
